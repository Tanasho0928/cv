%%%%%%%%%%%%%%%%%%%%%%%%%%%%%%%%%%%%%%%%%
% Medium Length Professional CV
% LaTeX Template
% Version 2.0 (8/5/13)
%
% This template has been downloaded from:
% http://www.LaTeXTemplates.com
%
% Original author:
% Rishi Shah 
%
% Important note:
% This template requires the resume.cls file to be in the same directory as the
% .tex file. The resume.cls file provides the resume style used for structuring the
% document.
%
%%%%%%%%%%%%%%%%%%%%%%%%%%%%%%%%%%%%%%%%%

%----------------------------------------------------------------------------------------
%	PACKAGES AND OTHER DOCUMENT CONFIGURATIONS
%----------------------------------------------------------------------------------------

\documentclass{resume} % Use the custom resume.cls style

\usepackage[left=0.75in,top=0.6in,right=0.75in,bottom=0.6in]{geometry} % Document margins
\usepackage[dvipdfmx]{graphicx}
\usepackage{multicol}
\usepackage[T1]{fontenc}
\newcommand{\tab}[1]{\hspace{.2667\textwidth}\rlap{#1}}
\newcommand{\itab}[1]{\hspace{0em}\rlap{#1}}
\name{Shohei Tanaka} % Your name
\address{(+81)9065939145 \\ tanaka.shohei.tj7@is.naist.jp} % Your phone number and email
\address{https://tanasho0928.github.io/}
\address{Gakuse-shukusha 1-210, Takayama-cho 8916-5, Ikoma-shi, Nara-ken, Japan, 630-0101} % Your address
%\address{123 Pleasant Lane \\ City, State 12345} % Your secondary addess (optional)

\begin{document}

%----------------------------------------------------------------------------------------
%	Summary
%----------------------------------------------------------------------------------------

\begin{rSection}{Summary}

I am a master student at Augmented Human Communication Lab (AHC), Nara Institute of Science and Technology (NAIST).
My research interests are dialogue system, event causality, and event representation.
Since I was an undergraduate student, I have been strongly interested in systems that communicate with users.
During my undergraduate, I made some products to pursue my interest.
I have been studying conversational dialogue systems in my graduate school.

\end{rSection}

%----------------------------------------------------------------------------------------
%	Work Experience
%----------------------------------------------------------------------------------------

\begin{rSection}{Work Experience}

\begin{rSubsection}{Research Assistant}{}{}{}
Nara Institute of Science and Technology \hfill Jun. 2019 -- Present
\\PRESTO, Japan Science and Technology Agency \hfill Apr. 2019 -- Present
\\NICT (National Institute of Information and Communications Technology) \hfill Nov. 2018 -- Mar. 2019
\end{rSubsection}

\begin{rSubsection}{Internships}{}{}{}
Fixstars Inc. \hfill Feb. 2018 -- Mar. 2018
\\Nissan Motor Co., Ltd. \hfill Aug. 2016 -- Sep. 2016
\end{rSubsection}

\end{rSection}

%----------------------------------------------------------------------------------------
%	Education
%----------------------------------------------------------------------------------------

\begin{rSection}{Education}

{\bf M.S. in Information Science} \hfill Mar. 2020 
\\ Nara Institute of Science and Technology
\\
\\{\bf B.S. in Engineering} \hfill 2018 
\\ Nagoya Institute of Technology
\\ {\bf Thesis:} A Design Methodology of Real-World Universal Voice Interface and its Model Case: "Easy Talk Suhama-Shoten"
\\ {\bf Supervisor:} Prof. Akinobu Lee

\end{rSection}

%----------------------------------------------------------------------------------------
%	Publications
%----------------------------------------------------------------------------------------

\begin{rSection}{Publications}

\begin{rSubsection}{International Conference Papers}{}{}{}
\begin{itemize}
\item \underline{Shohei Tanaka}, Koichiro Yoshino, Katsuhito Sudoh, Satoshi Nakamura.
\\"Conversational Response Re-ranking Based on Event Causality and Role Factored Tensor Event Embedding"
\\The 1st Workshop NLP for Conversational AI ACL 2019 Workshop (ConvAI), Florence, Italy, August 2019, Oral\&Poster, {\bf Best Paper Award}
\end{itemize}
\end{rSubsection}

\begin{rSubsection}{Domestic Conference Papers}{}{}{}
\begin{itemize}
\item \underline{Shohei Tanaka}, Koichiro Yoshino, Katsuhito Sudoh, Satoshi Nakamura.
\\"Conversational Response Selection Model Based on Event Coherency Estimation (in Japanese)"
\\The 87th Workshop of Special Interest Group on Spoken Language Understanding and Dialogue Processing (SIG-SLUD), Tokyo, Japan, December 2019, Poster

\item \underline{Shohei Tanaka}, Koichiro Yoshino, Katsuhito Sudoh, Satoshi Nakamura.
\\"Analysis of Conversational Response Re-ranking Based on Event Causality and Role Factored Tensor Event Embeding (in Japanese)"
\\The 241st Workshop of Special Interest Group on Natural Language (SIG-NL), Hokkaido, Japan, August 2019, Oral

\item \underline{Shohei Tanaka}, Koichiro Yoshino, Katsuhito Sudoh, Satoshi Nakamura.
\\"Incorporating Event Causality to Re-ranking for Conversational Dialogue Responses and its Evaluation (in Japanese)"
\\The 25th Annual Conference of The Association for Natural Language Processing (NLP), Nagoya, pp. 1026-1029, Japan, March 2019, Oral

\item \underline{Shohei Tanaka}, Koichiro Yoshino, Katsuhito Sudoh, Satoshi Nakamura.
\\"Incorporating Event Causality to Re-ranking for Conversational Dialogue Responses (in Japanese)"
\\The 84th Workshop of Special Interest Group on Spoken Language Understanding and Dialogue Processing (SIG-SLUD), Tokyo, Japan, November 2018, Poster

\item \underline{Shohei Tanaka}, Yusei Teramoto, Akinobu Lee.
\\"Easy Talk Suhama-Shoten (in Japanese)"
\\The 120th Workshop of Special Interest Group on Spoken Language Processing (SIG-SLP), Ibaraki, Japan, February 2018, Poster
\end{itemize}
\end{rSubsection}

\end{rSection}

%----------------------------------------------------------------------------------------
%	Awards
%----------------------------------------------------------------------------------------

\begin{rSection}{Awards}

\begin{itemize}
    \item Best Paper Award, The 1st Workshop NLP for Conversational AI ACL 2019 Workshop (ConvAI), 2019
    \item Scond Prize, Freestyle, App Development, NEXT COMMUNICATION AWARDS 2017
    \\(Japanese Smartphone Application Contest), 2017
\end{itemize}

\end{rSection}

%----------------------------------------------------------------------------------------
%	Skills
%----------------------------------------------------------------------------------------

\begin{rSection}{Skills}

\begin{multicols}{3}
\begin{itemize}
    \item Python
    \item PyTorch
    \item Swift
    \item C++
    \item Arduino
    \item Raspberry Pi
\end{itemize}
\end{multicols}

\end{rSection}

%----------------------------------------------------------------------------------------
%	Works
%----------------------------------------------------------------------------------------

\begin{rSection}{Works}

\begin{rSubsection}{Easy Talk Suhama-Shoten}{2017}{}{}
Suhama-Shoten is a spoken dialogue system to help elderly people with shopping.
Users can order items by only talking with a character. 
The system runs on a smartphone.
\\{\bf Demo Video (in Japanese):} \texttt{https://vimeo.com/239294559}
\\{\bf Slides (in Japanese):} \texttt{https://www.slideshare.net/ShoheiTanaka2/ncf2017-86057166}
\end{rSubsection}

% \begin{rSubsection}{Ama-no-Iwato Shiritori}{2016}{}{}
% {\em Ama-no-Iwato} is a story included in {\em Kojiki}, the Japanese oldest extant chronicle.
% {\em Shiritori} is a Japanese word game in which the players are required to say a word which begins with the final kana of the previous word.
% Users define a number and say words to make the character count equal to the defined number.
% \\{\bf Demo Video (in Japanese):} \texttt{https://youtu.be/9S4EGTc8tLk}
% \end{rSubsection}

\begin{rSubsection}{Spoken Controlled Car}{2016}{}{}
A spoken controlled toy car using Raspberry Pi and LEGO.
The commands are "foward," "back," "turn left," "turn right," and "come on."
The system was exhibited at Maker Faire Tokyo 2016.
\\{\bf Demo Video (in Japanese):} \texttt{https://youtu.be/7jeXsVbNgpM}
\end{rSubsection}

\begin{rSubsection}{Sea Lion Puppet}{2016}{}{}
A sea lion puppet for children using an Arduino.
By an iPhone application, users can make the puppet take actions: nodding, shaking arms, and making a sound.
\\{\bf Demo Video (in Japanese):} \texttt{https://youtu.be/aVoW5HBWl68}
\end{rSubsection}

\begin{rSubsection}{Pinbo (Penguin Puppet)}{2015}{}{}
A penguin puppet using an Arduino.
Reacting to sound and light, the puppet takes three actions: turning around, shaking hands, turning away.
\\{\bf Demo Video (in Japanese):} \texttt{https://youtu.be/NlB1CZMd\_CM}
\end{rSubsection}

\end{rSection}

% \newpage

% %----------------------------------------------------------------------------------------
% % Extra Curricular
% %----------------------------------------------------------------------------------------
% \begin{rSection}{Extra-Cirrucular} \itemsep -3pt
% \item Co-Organized “ Nirmitee 2017” - a National Symposium of Civil Department of MIT, Pune
% \item Attended a workshop on Autodesk Revit at IIT Bombay in 2014.
% \item Winner of Inter Departmental Football Competition 2015.
% \item Member of the  Rotaract Club Of Pune Pride from 2014 to 2017.
% \item Worked for a start-up company Named OUST as a Regional Marketing Manager
% %\item Trained and disciplined in National Cadet Corps (NCC), IIT Kanpur for a year.
%  %\item  Participated in Vijyoshi Camp 2012 organized at Indian Institute of Science, Bangalore.
%  %\item Won 2nd position in Kho-Kho in Intramurals conducted by Physical Education Section, IIT Kanpur.
%  %\item Pursued French as second language during secondary school from Grade 6 to Grade 10. Also participated in French Song Competition and French G.K. Quiz in Class 10th. %

% \end{rSection}

% \begin{rSection}{Personal Traits}
% \item Highly motivated and eager to learn new things.
% \item Strong motivational and leadership skills.
% \item Ability to work as an individual as well as in group.
% \end{rSection}
\end{document}
